\documentclass[letterpaper, 12pt]{article}
\usepackage[utf8]{inputenc}
\usepackage[german]{babel}
\usepackage{graphicx}
\usepackage{amsmath}
\usepackage{xcolor}
\usepackage{mathrsfs}
\usepackage{fancyhdr}
\usepackage{multirow}
\usepackage{amssymb}
\usepackage[a4paper]{geometry}
\geometry{top=1.8cm, bottom=1.8cm, left=1.5cm, right=1.5cm}

\pagestyle{fancy}
\lhead{Emiliano Guti\'errez Luengas}

\title{Problemas Fisicos}
\author{Emiliano Gutiérrez Luengas}
\date{Noviembre 2022}

\begin{document}

\maketitle

\begin{itemize}

    \item [a)]  Considerando un sistema en una dimensi\'on y sabiendo que $a= \frac{dv}{dt}$ y $v= \frac{dx}{dt}$, demuestre que la posici\'on se puede ver como:

\begin{center}
    $x = x_0 + v_0t + \frac{1}{2}at^2$
\end{center}

Sean $a= \frac{dv}{dt}$ y $v= \frac{dx}{dt}$, entonces tenemos que,

\begin{equation}
\label{second_derivative_displacement}
a= \frac{d}{dt} \cdot \frac{dx}{dt} = \frac{d^2x}{dt^2}
\end{equation} \\

Integramos ambos lados de la ecuaci\'on \ref{second_derivative_displacement} dos veces:

\begin{align*}
&\Rightarrow  \int a dt = \int \frac{d^2x}{dt^2} dt \\
&\Rightarrow at + c =\frac{dx}{dt} dt \\
&\Rightarrow  \int at + c dt = \int \frac{dx}{dt} dt \\
&\Rightarrow a \int t dt + c \int dt = \int \frac{dx}{dt} dt \\ 
\end{align*}

Esto da como resultado:

\begin{equation}
\label{Displacement_when_accelerating}
    \frac{1}{2}at^2 + ct + d = x
\end{equation}

Considerando que la funci\'on da como resultado una posici\'on, representada por la vairable 'x', es posible deducir que  c y d representan la velocidad y posici\'on inicial respectivamente. Por lo tanto, la ecuaci\'on \ref{Displacement_when_accelerating}:

\begin{center}
$\frac{1}{2}at^2 + ct + d = x \Rightarrow \frac{1}{2}at^2 + v_0t + x_0 = x$
\end{center}


    \item [b)] Considere una carrera entre dos coches, estos arrancando del reposo, mas el primer coche sale un segundo antes que el segundo. Los carros tienen una aceleraci\'on de $3.5 m\slash s^2$ y $4.9 m\slash s^2$ respectivamente. Encuentra:

    \begin{itemize}
        \item [i] En que momento el auto dos alcanza al auto uno:
        \item [ii] La posici\'on cuando el segundo alcance al primero:
        \item [iii] La velocidad para ambos carros en ese punto:
        \item [iv] 5 tiempos diferentes, 3 antes del tiempo cuando los carros se encuentran, y 2 despu\'es. Consecuentemente, realicen dos tablas, una para cada auto, con la siguiente informaci\'on; aceleraci\'on, tiempo, posici\'on y velocidad. \\
\end{itemize}

    Sea la posici\'on del auto uno definido por:

    \begin{equation}
    \label{Carro_uno}
        (x-x_0) = \frac{1}{2}a(t_1)^2 + v_1(t_1)
    \end{equation}

    y la posici\'on del auto dos:
      
    \begin{equation}
    \label{Carro_dos}
        (x-x_0) = \frac{1}{2}a(t_2)^2 + v_2(t_2)
    \end{equation}
    
Con $t_1 = t+1$ y $t_2 = t$. Igualmente, para determinar el momento en que el auto uno alcanza al auto dos, se debe de igualar la ecuaci\'on \ref{Carro_uno} a la ecuaci\'on \ref{Carro_dos}:

\begin{align*}
    &\Rightarrow \frac{1}{2}a(t_2)^2 + v_2(t_2) = \frac{1}{2}a(t_1)^2 + v_1(t_1) \\
    &\Rightarrow  \frac{1}{2}a(t)^2 + v_2(t) = \frac{1}{2}a(t+1)^2 + v_1(t+1)
\end{align*}

Reemplazando los valores para la aceleraci\'on, se obtiene que:

\begin{align*}
    &\Rightarrow  \frac{1}{2} 4.9(t)^2 + 0(t) = \frac{1}{2} 3.5(t+1)^2 + 0(t+1) \\
    &\Rightarrow \frac{4.9}{2}t^2 = \frac{3.5}{2}(t^2 + 2t + 1) \\
    &\Rightarrow \frac{4.9}{2}t^2 - \frac{3.5}{2}t^2 = 3.5t + \frac{3.5}{2} \\
    &\Rightarrow 1.4t^2 - 7t - 3.5 = 0 
\end{align*}

Aplicando la formula general para la resoluci\'on de ecuaciones de segundo grado se obtiene que:

\begin{center}
    $t' = 5.458$,  $t'' = -0.458$
\end{center}

Por deducci\'on, el tiempo tomado para que el segundo auto alcance al primero es de 5.458 segundos.

Una vez obtenido el valor de t, es facil determinar la posici\'on de los autos en aquel momento. Simplemente reemplazando en la ecuaci\'on \ref{Carro_dos}, se obtiene que:
 \begin{align*}
     &\Rightarrow (x-x_0) = \frac{1}{2}a(5.458)^2 + v_2(5.458) \\
     &\Rightarrow (x-x_0) = \frac{4.9}{2}\cdot 29.789764 + 0(5.458) \\
     &\Rightarrow (x-x_0) = 72.9849218
 \end{align*}

 Tomando en cuenta que la posici\'on inicial es la misma para cada veh\'iculo, o sea, cero, entonces:

 \begin{center}
     x = 72.9849218 m
 \end{center}

 Para obtener la velocidad, se deber\'an de derivar las ecuaciones \ref{Carro_uno} y \ref{Carro_dos}, que lleva a:

 \begin{equation}
 \label{velocidad_carro_uno}
     \frac{dx}{dt} = a(t_1) + v_1 = v_{c1}
 \end{equation}
 \begin{equation}
 \label{velocidad_carro_dos}
     \frac{dx}{dt} = a(t_2) + v_2 = v_{c2}
 \end{equation}

Reemplazando por t', se obtiene que:

\begin{align*}
   &\Rightarrow v_{c1} = a(t+1) + 0 \\
   &\Rightarrow v_{c1} = 3.5(6.458) + 0 \\
   &\Rightarrow v_{c1} = 22.603 \frac{m}{s}
\end{align*}

Y

\begin{align*}
   &\Rightarrow v_{c2} = a(t) + 0 \\
   &\Rightarrow v_{c2} = 4.9(5.458) + 0 \\
   &\Rightarrow v_{c2} = 26.7442 \frac{m}{s}
\end{align*}


Obteniendo los valores de la posici\'on y la velocidad  del auto 1, para los tiempos 0, 1, 2, 6 y 7, se obtiene que: \\


\begin{tabular}{|cllll|}
\hline
\multicolumn{5}{|c|}{Auto 1}                                                                                                                              \\ \hline
\multicolumn{2}{|l|}{Variables No Dependientes del Tiempo} & \multicolumn{3}{l|}{Variables Dependientes del Tiempo}                                       \\ \hline
\multicolumn{2}{|c|}{Aceleración}                          & \multicolumn{1}{c|}{Tiempo} & \multicolumn{1}{c|}{Posición} & \multicolumn{1}{c|}{Velocidad} \\ \hline
\multicolumn{2}{|c|}{\multirow{5}{*}{3.5}}                 & \multicolumn{1}{l|}{0}      & \multicolumn{1}{l|}{1.75}     & 3.5                            \\ \cline{3-5} 
\multicolumn{2}{|c|}{}                                     & \multicolumn{1}{l|}{1}      & \multicolumn{1}{l|}{7}        & 7                              \\ \cline{3-5} 
\multicolumn{2}{|c|}{}                                     & \multicolumn{1}{l|}{2}      & \multicolumn{1}{l|}{15.75}    & 10.5                           \\ \cline{3-5} 
\multicolumn{2}{|c|}{}                                     & \multicolumn{1}{l|}{6}      & \multicolumn{1}{l|}{85.75}    & 24.5                           \\ \cline{3-5} 
\multicolumn{2}{|c|}{}                                     & \multicolumn{1}{l|}{7}      & \multicolumn{1}{l|}{112}      & 28                             \\ \hline
\end{tabular} \\

De igual manera, obteniendo los valores de la posici\'on y la velocidad  del auto 2 para los tiempos 0, 1, 2, 6 y 7, se obtiene que: \\

\begin{table}[h]
\begin{tabular}{|cllll|}
\hline
\multicolumn{5}{|c|}{Auto 2}                                                                                                                              \\ \hline
\multicolumn{2}{|l|}{Variables No Dependientes del Tiempo} & \multicolumn{3}{l|}{Variables Dependientes del Tiempo}                                       \\ \hline
\multicolumn{2}{|c|}{Aceleración}                          & \multicolumn{1}{c|}{Tiempo} & \multicolumn{1}{c|}{Posición} & \multicolumn{1}{c|}{Velocidad} \\ \hline
\multicolumn{2}{|c|}{\multirow{5}{*}{4.9}}                 & \multicolumn{1}{l|}{0}      & \multicolumn{1}{l|}{0}        & 0                              \\ \cline{3-5} 
\multicolumn{2}{|c|}{}                                     & \multicolumn{1}{l|}{1}      & \multicolumn{1}{l|}{2.45}     & 4.9                            \\ \cline{3-5} 
\multicolumn{2}{|c|}{}                                     & \multicolumn{1}{l|}{2}      & \multicolumn{1}{l|}{9.8}      & 9.8                            \\ \cline{3-5} 
\multicolumn{2}{|c|}{}                                     & \multicolumn{1}{l|}{6}      & \multicolumn{1}{l|}{88.2}     & 29.4                           \\ \cline{3-5} 
\multicolumn{2}{|c|}{}                                     & \multicolumn{1}{l|}{7}      & \multicolumn{1}{l|}{120.05}   & 34.3                           \\ \hline
\end{tabular}
\end{table}


 
    \item [c)] Considere el siguiente sistema: dos bloques de masa $m_1$ y $m_2$ estan unidos por una cuerda ideal y descanzan sobre una superficie horizontal sin roce. Si una fuerza de magnitud A se le aplica al bloque de la masa $m_2$ horizontalmente, en la direcci\'on opuesta a la posici\'on del bloque de masa $m_1$, realice los respectivos diagramas de cuerpo libre y an\'exelo con una imagen. A partir de estos, determine la aceleraci\'on del sistema y la tensi\'on de la cuerda entre los bloques. \\

\begin{figure}[h]
    \centering
    \includegraphics[scale=0.12]{IMG-1445.jpg}
    \caption{Diagrama de Cuerpo Libre para dos Bloques}
    \label{Diagrama_cuerpo_libre}
\end{figure}

El sistema puede ser dividido en dos diagramas de cuerpo libre, facilitando los calculos \textit{a priori}. De esta forma, se puede ver que la aceleraci\'on del segundo bloque es:

\begin{center}
    $a_2 = \frac{A}{m_2}$
\end{center}

y, 

\begin{center}
    $a_1 = \frac{T}{m_1}$
\end{center}

De igual forma, se tiene que:

\begin{center}
    $A - T = \sum F = m_s \cdot a_s$
\end{center}

Donde, 
\begin{center}
    $m_s = (m_1 + m_2)$, $a_s = a_1 + a_2$
\end{center}

De forma que,

\begin{center}
    $A - T = (m_1 + m_2)\cdot (a_1 + a_2)$
\end{center}

\begin{center}
    $\therefore T = A - ((m_1 + m_2)\cdot (a_1 + a_2))$
\end{center}

\end{itemize}
\end{document}
