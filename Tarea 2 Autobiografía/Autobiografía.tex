\documentclass[letterpaper, 12pt]{article}
\usepackage[utf8]{inputenc}
\usepackage[spanish]{babel}
\usepackage[usenames]{color}

\title{Autobiografía de un joven f\'isico}
\author{Emiliano Gutiérrez Luengas}
\date{11 de Septiembre, 2022}

\begin{document}

\maketitle

\newpage
\section{\Large{Academia}}
  \subsection{\large{Pasado}}
   Vengo del colegio Peterson, el cual quedaba de una hora a dos de mi casa. La ruta en cuesti\'on era: por el eje 5 desde la Rep\'ublica Federal hasta avenida V\'ia Lactea. De ah\'i, se tomaba la avenida Observatorio hasta la salida a la carretera libre M\'exico Toluca. Finalmente, desde la avenida Carlos Ech\'anove hasta Huizachito.


  \subsection{\large{Actualidad}}
   Actualmente, la facultad sigue qued\'andome lejos de mi hogar, aproximadamente de una hora a una hora y media. La ruta es la siguiente: por el eje 5 como en la ruta anterior, hasta el Perif\'erico Oriente. subsecuentemente, se sigue hasta avenida Insurgentes.


  \subsection{\large{¿Por qu\'e f\'sicia?}}
  La f\'sica ha estado presente en mi vida desde la secundaria, y me vi inmerso en la inmensidad del misterio que lo rodeaba cuando le\'i, en clase de f\'isica, un art\'iculo sobre la gravedad. Desde ese momento, me decid\'i a estudiarla. En cualquier caso, desde peque\~no la ciencia era lo \'unico presente en mi mente, por lo que era inevitable que buscara una carrera en f\'isica. Si tratara de explicarlo con mayor detalle, supongo que no podr\'ia. Ser\'ia como tratar de explicar porqu\'e uno busca comer cuando tiene hambre o dormir cuando tiene sue\~no: asumo que es simplemente mi naturaleza.

\newpage
\section{\Large{Hobbies}}
  \subsection{\large{Saxof\'on}}
  Siendo sincero, no suelo pensar porqu\'e me gusta lo que hago, pero si fuera a dar una raz\'on principal (en el caso del saxo\'on) ser\'ia porque disfruto de tocar m\'usica, porque la sensaci\'on de las notas emvolvi\'endote cuando las tocas con cuerpo es \'unica. Debido a la naturaleza del saxof\'on como un instrumento celoso, debo de practicar esencialmente todos los d\'ias. Claro que a veces esto no pasa, pero mientras el lapso de tiempo entre pr\'acticas sea menor a una semana, no deber\'ia de haber ning\'un problema.


  \subsection{\large{Tae Kwon Do}}
  El Tae Kwon Do lo llevo practicando desde que tengo memoria, y como siempre me han interesado las artes marciales, pues me termin\'e por envolver totalmente en el mundo de este arte coreano. Lo disfruto bastante, y desconozco si es por la adrenalina al momento de combatir contra un oponente igual o superior en t\'erminos de habilidad, o el ver como tu t\'ecnica mejora con cada pr\'actica, pero al final simplemente me veo atraido.

\newpage
\section{\Large{M\'usica preferida}}
  \subsection{\large{M\'usica Cl\'asica}}
  Principalmente suelo escuchar m\'usica cl\'asica cuando me place, aunque prefiero escucharla cuando estoy solo en mi cuarto, o haciendo alguna actividad. De este g\'enero, dir\'ia que dos ejemplos magistrales ser\'ian: los conciertos de piano en B-plano menor, Opus 23 de Tchaikovsky, y "\textit{le quattro stagioni}" de Antonio Vivaldi.


  \subsection{\large{Hip-Hop Rap}}
  Llego a escucharlo con una mayor frecuencia que la m\'usica cl\'asica, mas escucho el g\'enero en las mismas circunstancias. Ejemplos de mis canciones preferidas del antedicho ser\'ian: las canciones pertenecientes al album "\textit{Dimensiones y cuerdas}" de Ozelot ("\textit{El monstruo}, "\textit{luz}", "\textit{El ciclo del carbono}", "\textit{Por qu\'e existe algo en lugar de nada}") y "\texit{Mi determinaci\'on}" de Solitario.

\newpage
{\section{\Large{Lecturas favoritas}}
  \subsection{\large{Textos "fuera de lo com\'un"}}
  \textbf{Personalmente suelo preferir textos antiguos o de caracter filos\'ofico, tales como (en el caso de los textos antiguos) "\textit{La Iliada}" de Homero, "\textit{La Rep\'ublica}" de Plat\'on o, uno m\'as reciente, "\textit{La Divina Comedia}" de Dante Alighieri.}\\

  \textcolor{cyan}{Por otro lado, con respecto a textos filos\'oficos, me gusta leer a Nietzsche. Termin\'e de leer \textsl{Also sprach Zaratustra} en preparatoria y, actualmente, estoy leyendo \textit{La genealog\'ia de la moral} del mismo. Su filosof\'ia me parece interesante y, aunado a que mi t\'io me introdujo al autor a la edad de 12 (aproximadamente), resulto en un gusto por lecturas densas.}
  
  \subsection{\large{Mangas}}
  \textcolor{red}{En el caso del manga, se podr\'ia decir que empec\'e como todo buen fan\'atico: viendo \textit{Dragon Ball Z}. Durante la pandemia opt\'e por leer mangas (algo que no hac\'ia), debido al aburrimiento. Entre los que he le\'ido han sido: \textit{Hajime no Ippo, Vinlando Saga, HunterXHunter, Grand Blue, Magi}, etc. Para contextualizar, suelo usar mi tel\'efono un promedio de 6 horas, de las cuales un 40\% se emplea en safari. En cualquier caso, leer manga es una de mis actividades preferidas.}

  
  \subsubsection{\large{Berserk}}
  Concluyendo con la secci\'on, he de mencionar una de mis obras favoritas, debido al desarrollo de los personajes, del mundo y las consecuencias de los actos del protagonista. Es una obra sublime, en la que incluso el aumento de poder del principal es realista, congruente y acarrea problemas fundamentales. Empero, esta parte ser\'a dedicada al antagonista principal de la obra: el antiguo lider de la banda del Halc\'on y m\'axmo traidor griffith.\\

  Siendo m\'as espec\'ifico, me dedicar\'e a hablar de sus acciones y si lo que hizo fue bueno o malo. Primeramente, Griffith era la figura cuasi-divina que todos dentro del grupo adoraban, que ir\'onicamente se mostr\'o como el m\'as humano que todos los personajes. A diferencia de Guts, el protagonista, el cual es representado como un monstruo que puede matar a m\'as de cien soldados mientras est\'a herido, Griffith es un personaje fragil, siendo codiciado por nobles e impotente ante el mundo que lo rodea; es un prisionero de las reglas del mundo, de la sociedad y de sus propios deseos. Por otro lado Guts, en su momento era libre, contrastando con enormemente con el inicio de su vida (abandonado recien nacido, abusado por sus compa\~neros adultos y forzado por Griffith a meterse en la banda del Halc\'on). Por este hecho, Griffith se vuelve en un personaje tr\'agico, que por su impotencia decide, sin vacilar, sacrificar a sus compa\~neros de a\~nos para alcanzar un mayor poder.\\

  En este punto, los papeles tanto de villano como de protagonista se invierten. Guts queda cegado por su ira, volvi\'endose esclavo del deseo de vengar a la banda del Halc\'on, mientras que Griffith construye su propio reino y reune un nuevo s\'equito de \textit{Ap\'ostoles} para comandar. Bas\'andonos en lo presentado, se puede observar que Griffith es deplorable como ser pensante, pero a\'un no acaba la cosa: sigue teniendo el descaro de presentarse ante los sobrevivientes de la anterior banda para pedirles que se unan a \'el, tal como pas\'o en el \'ultimo cap\'itulo, llev\'andose a Casca (protagonista secundaria de la obra), mientras que destruye el reino de los elfos, ocasionando que las criaturas m\'agicas desaparezcan del mundo. Es una basura, lo que a su vez lo vuelve un personaje excelente. Realmente logra, por como lo presenta el autor, manifestar en el lector un odio inmenso. Finalizando y respondiendo a la pregunta inicial: las acciones de Griffith no pueden ser justificadas y, mucho menos, puede uno como lector empatizar.
\end{document}
